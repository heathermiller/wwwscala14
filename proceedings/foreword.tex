% \documentclass[a4paper]{memoir}
\documentclass[10pt]{book}

\usepackage{xspace}
\usepackage{graphicx}
\usepackage[normalem]{ulem} % \emph should italicize, not underline
\usepackage{microtype}
\usepackage{tocloft}
\usepackage{times}

\usepackage{etoolbox}% http://ctan.org/pkg/etoolbox
\makeatletter
% \patchcmd{<cmd>}{<search>}{<replace>}{<success>}{<failure>}
\patchcmd{\@makechapterhead}{\huge}{\large}{}{}% for \chapter
\patchcmd{\@makechapterhead}{\Huge}{\large}{}{}% for \chapter
\patchcmd{\@makeschapterhead}{\Huge}{\large}{}{}% for \chapter*
\makeatother

\usepackage[margin={1.5in,1in}]{geometry}

\title{}
% \subtitle{4th Workshop on Scala with ECOOP'13, July 2nd 2013, Montpellier, France}

\date{}

\newcommand{\papertitle}[2]{\noindent \textbf{#1} \dots #2\\}
\newcommand{\paperauthor}[1]{by #1\\[2em]}

% \renewcommand*{\titlepagestyle}{empty}

\pagestyle{empty}
\usepackage{hyperref}
\hypersetup{
   a4paper,
   colorlinks,
   urlcolor=blue,
   citecolor=blue,
   linkcolor=blue,
   pdfauthor = {Damien Cassou},
}

\begin{document}
\thispagestyle{empty}

% \maketitle

{\centering \LARGE \bf Chairs' Welcome\par}

\vspace{0.5cm}
\normalsize

\noindent
It is our great pleasure to welcome you to the Scala Workshop 2014. The
meeting follows in the tradition of 4 previous Scala Workshops. The Scala
Workshop 2014 is co-located with the 28th edition of the European Conference
on Object-Oriented Programming (ECOOP).
\\

\noindent
Scala is a general-purpose programming language designed to express common
programming patterns in a concise, elegant, and type-safe way. It smoothly
integrates features of object-oriented and functional languages. The Scala
Workshop is a forum for researchers and practitioners to share new ideas and
results of interest to the Scala community.
\\

\noindent
This edition embraces elements of the format introduced with the 2013 edition,
such as academic student talks, which are not accompanied by papers. Student
talks are about 5-10 minutes long, presenting ongoing or completed research
related to Scala. The 2014 edition of the Scala Workshop further innovates on
the format by introducing a new category of ``Open Source Talks.'' These are
short talks about open-source projects using Scala presented by contributors
to these projects. Like student talks, open source talks are not accompanied
by papers.
\\

\noindent
This year's call for papers attracted 14 submissions of regular research
papers. Each of the papers was reviewed by at least 3 Program Committee
members. During a week-long electronic meeting, the Program Committee selected
9 papers for publication in these proceedings and for presentation at the
workshop. In addition, the committee accepted 2 papers exclusively for
presentation. Furthermore, the call for papers attracted 9 submissions of
student talks. The organizers selected 7 student talks for presentation at the
workshop.
\\

\noindent
Finally, the program includes a panel discussion of international experts on
``Scala and Next-Generation Languages: Language Design for Mainstream Software
Engineering.''
\\

\noindent
Many people have helped make the Scala Workshop 2014 a reality. We would like
to thank all the authors of the submitted papers, and the Program Committee
for their reviews, thoughtful discussion, and helpful feedback to the authors.
We are very grateful for the fruitful collaboration with ECOOP. Special thanks
go to Tobias Wrigstad, the ECOOP Organizing Chair, and to Nate Nystrom, the
ECOOP Workshop Chair. Tobias helped us organize our sponsorship program and
our student grant scheme, both of which were exceptionally successful this
year. Nate helped us with the rest of the workshop organization, including
these workshop proceedings. Doug Lea and Martin Odersky, who served on the
Organizing Committee, provided valuable insight throughout the organization
of this edition of the Scala Workshop. The EasyChair conference management system
enabled the reviewing  process. Finally, thanks to Typesafe, SoundCloud, Goldman
Sachs, and innoQ for sponsoring the workshop.
\\

\begin{tabular}{ll}
{\bf Heather Miller}              & {\bf Philipp Haller}\\
{\em Scala'14 Program Co-Chair}   & {\em Scala'14 Program Co-Chair}\\
{\em EPFL, Switzerland}           & {\em Typesafe Switzerland}\\
\end{tabular}

\end{document}

%%% Local Variables:
%%% mode: latex
%%% TeX-master: t
%%% End:
